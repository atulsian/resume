\documentclass{resume}
\usepackage{url}
\usepackage[pdfborderstyle={/W 0}]{hyperref}

%
\renewcommand{\labelitemi}{$\diamond$}
\newcommand{\teamsize}{\\\sc\footnotesize Team Size: }


\author{Ashish Tulsian}


% - Address --------------------------------------------------------
\address {
    \href{mailto:ashish.tulsian@gmail.com}{ashish.tulsian@gmail.com}\\
    \href{mailto:ashish_tulsian@rediffmail.com}{ashish\_tulsian@rediffmail.com}\\
}
{
    +91 990 021 3120
}


\begin{document} \maketitle

% - Education ---------------------------------------------------
\begin{category}{Education}{}

    \item {\topic M.Tech in Information Technology}
        {\footnotesize(3.77/4.00 CGPA)}
        {\period Aug 2004 - Jun 2006}
        \begin{quote}
            International Institute of Information Technology, Bangalore
        \end{quote}

\end{category}


% - Experience --------------------------------------------------
\begin{category}{Experience}{}

    \item {\topic Software Engineer,} Fiberlink India Pvt. Ltd., Bangalore
        {\period Jul 2006 - Dec 2008}
        \begin{quote}
            Fiberlink is a trusted mobility expert and the world's foremost
            provider of {\em Mobility as a Service} (MaaS)---on-demand services that help
            enterprises connect, control and secure laptops and mobile devices.
        \end{quote}
        \begin{quote}
            Was involved in the development of different products such as
            Extend360, Fiberlink System Probe and E360 Management Agent; also
            involved in the development of components of Fiberlink's MaaS
            (Mobility as a Service) Platform
        \end{quote}

    \item {\topic Intern,} Intel India Technologies Ltd., Bangalore
        {\period Jan 2006 - Jun 2006}
        \begin{quote}
            Worked for the development of a simulator for next generation
            graphics shader model. Contributed in the design as well as the
        implementation of the simulator.
        \end{quote}

\end{category}


% - Projects ---------------------------------------------------
\begin{category}{Selected Projects}{}

    \item {\topic Extend360 Management Agent,} Fiberlink
        {\period Jan 2008 - Sep 2008}
        {\teamsize 4}
        \begin{quote}
            Extend360 Management Agent (EMA) is a control and visibility
            management agent, which is responsible for managing various mobile
            workers of an enterprise. It provides a management UI, which gives
            control to the IT admins to manage endpoints irrespective of
            whether a user is in the corporate network or not.

            \begin{itemize}
                \item Designed and implemented a cron based scheduling framework.
                \item Designed and implemented a system tray component for EMA.
            \end{itemize}
        \end{quote}
        \begin{quote}
            The extended scope of the project involved a {\hl platform independence}
            exercise.

            \begin{itemize}
                \item Converted string objects to platform independent
                    variants.
                \item Created a platform independent timer class.
                \item Created a time object that can be used across platforms.
            \end{itemize}
        \end{quote}

    \item {\topic Fiberlink System Probe,} Fiberlink
        {\period Jan 2007 - Dec 2007}
        {\teamsize 2-3}
        \begin{quote}
            Fiberlink System Probe (FSP) runs as an ActiveX control on a user's system
            and is designed to efficiently collect and report customer end
            point and environment configuration information.

            \begin{itemize}
                \item Was involved from the inception of the project and was the
                    lead developer for the product.
                \item Performed an in depth study of the functioning of an
                    in-house framework and built an ActiveX framework over it.
                \item Developed DLL based agents to collect the customer's
                    environment information.
            \end{itemize}
        \end{quote}
        \begin{quote}
            MaaS Development.
            \begin{itemize}
                \item Designed and developed a MaaS component which has the
                    ability to run an ActiveX control.
                \item Developed a reporting framework which enables the user to
                    view the collected information.
                \item Added workflows to view, add, manage and delete user's
                    machine images.
                \item Added mappings and workflows to save the collected
                    information using the in-house BOMS framework.
            \end{itemize}
        \end{quote}

    \pagebreak

    \item {\topic Charlotte,} Fiberlink
        {\period Oct 2008 - Dec 2008}
        {\teamsize 4}
        \begin{quote}
            Charlotte is an inspection agent that is able to discover the
            environmental attributes of a particular endpoint and evaluate them
            to more quickly identify and diagnose potential issues for a
            variety of applications. It is the next generation of Fiberlink
            System Probe (FSP) and uses Extend360 Management Agent (EMA) as its
            core engine.
            \begin{itemize}
                \item Was involved in integration of FSP ActiveX with EMA core.
                \item Developed Lua scripts and wrote Lua extensions to collect
                    customers' environment information and identify potential
                    issues.
            \end{itemize}
        \end{quote}

    \item {\topic Valentina to SQLite Migration for Extend360,} Fiberlink
        {\period Aug 2006 - Dec 2006}
        {\teamsize 2}
        \begin{quote}
            Extend360 is Fiberlink's flagship product. It is a security and
            access management software which resides on a client machine and is
            controlled by a platform which is a hosted service.
        \end{quote}
        \begin{quote}
            E360 stores information like policies, phonebook, profiles etc in a
            Valentina database on client machine.
            \begin{itemize}
                \item Was involved in migration of the Valentina database to a
                    SQLite database.
                \item Was involved in changing the adhoc database access management
                    to a more organized DAL based model.
            \end{itemize}
        \end{quote}

    \item {\topic Malm\"o,} IIIT, Bangalore %%%% Todo: Better name
        {\period Aug 2005 - Dec 2005}
        {\teamsize 3}
        \begin{quote}
            A Relational Database Management System supporting a subset of the
            standard SQL. The system supports indexing of records and has its
            own buffer management to increase the access speed.
        \end{quote}
        \begin{quote}
            Contributed to
            the design and was responsible for implementing the Database Engine
            and Parser.
         \end{quote}

    \item {\topic Desktop Search Engine,} IIIT, Bangalore
        {\period Sep 2004 - Oct 2004}
        {\teamsize 3}
        \begin{quote}
            The project involved building a Desktop Search Engine which can
            handle phrase matching and stemming. Designed and implemented the indexer.
        \end{quote}

\end{category}


% - Skills --------------------------------------------------
\begin{category}{Skills}{}

    \item {\topic Languages:} C, C++, Java, Lua, JavaScript
    \item {\topic Databases:} MySQL, MS SQL, Oracle
    \item {\topic Tools:} Visual Studio,  Eclipse, Toad, Subversion, Memory
        Validator
    \item {\topic Other:} JSP, HTML, XML, Spring, Tomcat,  ActiveX, COM

\end{category}


% - Honours ---------------------------------------------------
\begin{category}{Honours}{$\diamond$}

    \item Awarded a scholarship under HP Shiksha program for academic excellence at
        IIIT, Bangalore.
    %{\period Jun 2006}

    \item Filed a patent for the game of inverse chess (Chess played backwards
        in time).
    %{\period Aug 2006}

    \item Presented poster on ``Data centric ad hoc networks'' at Intel India
        Student Research Contest 2005-06.
    %{\period Nov 2005}

\end{category}


\end{document}
