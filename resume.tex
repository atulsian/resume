\documentclass{resume}
\usepackage{url}
\usepackage[pdfborderstyle={/W 0}]{hyperref}

%
% set the space used for category titles here:
% use the same value for oddsidemargin and marginparwidth [the latter 
%         will be reset to account for marginparsep]
% 
\setlength{\oddsidemargin}{1in}
\setlength{\marginparwidth}{1in}

% 
% calculate other dimensions [textwidth and evensidemargin] 
% in function of oddsidemargin and marginparwidth: 
% would be nicer to put in the class file...
%
\addtolength{\marginparwidth}{-\marginparsep}
 \setlength{\evensidemargin}{\oddsidemargin}
 \setlength{\textwidth}{\paperwidth}
 \addtolength{\textwidth}{-2in}
 \addtolength{\textwidth}{-2\oddsidemargin}
 \addtolength{\textwidth}{\marginparwidth}
 \addtolength{\textwidth}{\marginparsep}

%
\setlength{\topmargin}{-1in}

%
\renewcommand{\titlenamefont}{\bf\Large}
\renewcommand{\categoryfont}{\sc}
\renewcommand{\labelcitem}{}

%\newcommand{\period}{\hspace{-0.35em}, \small\sl}
\newcommand{\period}{\hfill\small\sl}


\author{Siddhartha Reddy Kothakapu}


% - Address --------------------------------------------------------
\address
{
    \url{http://www.grok.in/}\\
    \url{http://www.linkedin.com/in/sidsr}
}
{
    \href{mailto:sids@grok.in}{sids@grok.in}\\
    +91 998 028 7437
}


\begin{document}
\maketitle

% - Education ---------------------------------------------------
\begin{category}{Education}

    \citem{M.Tech in Information Technology} {\footnotesize(3.71 CGPA)} {\period Jun 2006}
    \begin{quote}
        International Institute of Information Technology, Bangalore
    \end{quote}

\end{category}


% - Experience --------------------------------------------------
\begin{category}{Experience}

    \citem{Senior Software Engineer,} Ziva Software, Bangalore {\period Dec 2007 - Present}
    \begin{quote}
        Blah
    \end{quote}

    \citem{Software Engineer,} Ziva Software, Bangalore {\period Jul 2006 - Nov 2007}
    \begin{quote}
        Was the first non-founding employee to join the company. Have been core member, involved in most of the projects and actively participated in defining the development roadmap for them.
    \end{quote}

\end{category}


% -Publications ------------------------------------------ %
\begin{category}{Publication}

    \citemnobullet {\bf Measures of ``Ignorance'' on the Web} {\period Dec 2006}
    \begin{quote}
        International Conference on Management of Data, Delhi, India
    \end{quote}

\end{category}


% - Projects ---------------------------------------------------
\begin{category}{Selected Projects}

    \citem{Buddhi Online,} Ziva Software, Bangalore {\period Nov 2007 - Present}
    \begin{quote}
        Prototype of a system to fetch {\em Yahoo! BOSS} Web search results at run time, fetch the web pages in the results and extract {\em snippets} that can possibly provide an answer to the query right away.
    \end{quote}

    \citem{Buddhi,} Ziva Software, Bangalore {\period Apr 2007 - Present}
    \begin{quote}
        A dream to crawl the Web and automagically extract {\em information content entities} off the crawled pages.
    \end{quote}
    \begin{itemize}
        \item Designed and implemented
            \begin{itemize}
                \item a {\em Focused Crawler} based on the {\em Nutch} open source crawler.
                \item a {\em HTML reducer} that transforms complex HTML pages into minimal HTML for easier analysis.
                \item a {\em page abstractor} that discerns logical structures (such as tables, lists etc.) from unstructured HTML.
            \end{itemize}
        \item Set up the {\em Hadoop} based infrastructure for the crawling and extraction.
            This was initially set up on a cluster of {\em Amazon EC2} machines with {\em Amazon S3} as a data store.
    \end{itemize}

    \citem{Zook,} Ziva Software, Bangalore {\period Jan 2007 - Present}
    \begin{quote}
        Ziva Software's flagship mobile search engine. The main promise is to provide exact answers to questions in the least number of clicks; implying that the search engine provides information right away rather than directing the user to another website that {\em might} have the information.
    \end{quote}
    \begin{itemize}
        \item Set up the search system infrastructure.
        \item Designed the {\em User Experience} and {\em User Interface} of the WAP and SMS interfaces.
        \item Responsible for deploying releases.
        \item Created a HTTP based API for accessing the search system.
    \end{itemize}

    \citem{Better Web Buys} {\period Feb 2006 - May 2006}
    \begin{quote}
        A {\em faceted price comparison shopping search engine} with over 20 million items across more than 100 merchants. Was completely responsible for the whole project right from downloading feeds, parsing them, storing them in a database, indexing, searching, user experience and setting up the entire system on a cluster of machines. Used {\em Sphynx} as the backend search system.
    \end{quote}

    \citem{yadb --- Yet Another DataBase,} IIIT, Bangalore {\period Aug 2005 - Dec 2005}
    \begin{quote}
        A reasonably {\em complete RDBMS} with a buffered memory manager, B+ tree indexing and a complete database management module including support for basic SQL, table management, paged storage etc. Part of the {\em Advanced Databases} course.
    \end{quote}

    \citem{Top News,} IIIT, Bangalore {\period Sep 2005 - Oct 2005}
    \begin{quote}
        A system to bring out the top news of the hour by clustering the news items obtained from the feeds provided by several news sources. Any news being discussed by multiple sources within a given (variable) timeframe is marked as top news. A bi-partite graph based clustering algorithm was used for clustering based on just the title and summary of the news item. {\em Part of the Web Information Retreival} course.
    \end{quote}

    \citem{SOS --- S Operating System} {\period Sep 2002 - Oct 2003}
    \begin{quote}
    \end{quote}

\end{category}


% - Skills --------------------------------------------------
\begin{category}{Skills}

    \citem{Languages:} Perl, Java, C/C++, Ruby, Python

    \citem{Softwares/Frameworks:} Hadoop, Nutch, Lucene, Solr, Sphynx, WEKA, Ruby on Rails, Prototype, Hibernate

\end{category}


% - Intersts --------------------------------------------------
\begin{category}{Interests}
    \citemnobullet Information Engineering---Retrieval, Extraction \& Management, Scalable Architectures, Distributed \& Cloud Computing, Usability \& User Experience
\end{category}


\renewcommand{\labelcitem}{$\diamond$}


% - Honors ---------------------------------------------------
\begin{category}{Honors}
    \citembullet Valedictorian for the class for 2006 at IIIT, Bangalore. {\period Jun 2006}
    \citembullet Was awarded the Honeywell Scholarship for excellence at IIIT, Bangalore. {\period Jun 2006}
\end{category}


\end{document}
